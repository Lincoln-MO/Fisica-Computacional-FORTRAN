\chapter{Grupo de renormaliza\c{c}\~ao}
\label{RG}
\noindent

\section{Blocos de spins}
\noindent

Nós aprendemos que as propriedades críticas de um sistema não dependem dos detalhes de curta distância, mas somente da natureza das flutuações de longos comprimentos de onda. Isso sugere que alugém poderia com os graus de liberdade irrelevantes po um procedimento de granulamento grosso (através do qual os detalhes de uma escala atômica são tirados de médias) oara todas escalas de grandes distâncias, até alcançar o comprimento de correlação.

Kadanoff \footnote{L. P. Kadanoff, {\it Physics} {\bf 2}, 263 (1966).} primeiro introduziu a idéia em termos de transformações de blocos de spins em modelos de Ising.

\section{O modelo de Ising unidimensional}
\noindent

\section{Transforma\c{c}\~ao do Grupo de Renormaliza\c{c}\~ao}
\noindent

\section{Pontos fixos e campos de escala}
\noindent

\section{Formula\c{c}\~ao do espa\c{c}o de momentos}
\noindent

\section{O modelo gaussiano}
\noindent

\section{O modelo de Landau-Wilson}
\noindent