\chapter{As leis da termodin\^amica}
\noindent

\section{Conceitos preliminares}
\noindent
Termodin\^amica \'e uma teoria fenomenol\'ogica da mat\'eria. Dessa forma, ela desenha os seus conceitos diretamente dos experimentos. A seguir existe uma lista de alguns conceitos de trabalho os quais o f\'isico, atrav\'es da experi\^encia, achou ser conveniente introduzir. N\'os seremos extremamente breves, pois o leitor \'e supostamente familiarizado com esses conceitos.
\begin{itemize}
\item[(a)] Um {\it sistema termodin\^amico} \'e qualquer sistema macrosc\'opico.
\item[(b)] {\it Par\^ametros termodin\^amicos} s\~ao quantidades macrosc\'opicas mensur\'aveis associadas com o sistema, tais como a press\~ao $P$, o volume $V$, a temperatura $T$, e o campo magn\'etico $H$. Elas s\~ao definidas experimentalmente.
\item[(c)] Um {\it estado termodin\^amico} \'e especificado por um conjunto de valores de todos os par\^ametros termodin\^amicos necess\'arios para a descri\c{c}\~ao do sistema.
\item[(d)] {\it Equil\'ibrio termodin\^amico} prevalece quando o estado termodin\^amico de um sistema n\~ao muda com o tempo.
\item[(e)] A {\it equa\c{c}\~ao de estado} \'e uma rela\c{c}\~ao funcional entre os par\^ametros termodin\^amicos para um sistema em equil\'ibrio. Se $P$, $V$ e $T$ s\~ao os par\^ametros termodi\^amicos de um sistema, a equa\c{c}\~ao de estado assume a forma
\begin{center}
$f(P,V,T)=0$
\end{center}
a qual reduz o n\'umero de vari\'aveis independentes do sistemas de tr\^es para duas. A fun\c{c}\~ao $f$ \'e supostamente dada como parte da especifica\c{c}\~ao do sistema. \'E costumeiro representar o estado de um sistema por um ponto num espa\c{c}o $P$-$V$-$T$ tridimensional. A equa\c{c}\~ao de estado ent\~ao define uma superf\'icie nesse espa\c{c}o, como mostrado na Fig. \ref{figC01_001}. Qualquer ponto sobre essa superf\'icie representa um estado em equil\'ibrio. Na termodin\^amica um estado automaticamente significao um estado em equil\'ibrio a menos que o contr\'ario seja especificado.
\begin{figure}[h]
\begin{center}
%\includegraphics[angle=0,width=0.6\textwidth,clip]{fig001.eps}
\renewcommand{\figurename}{Fig.}
\caption{Representa\c{c}\~ao geom\'etrica da equa\c{c}\~ao de estado.}
\label{figC01_001}
\end{center}
\end{figure}
\item[(f)] Uma {\it transforma\c{c}\~ao termodin\^amica} \'e uma mudan\c{c}a de estado. Se um estado inicial \'e um estado de equil\'ibrio, a transforma\c{c}\~ao pode ser feita sobre somente por mudan\c{c}as nas condi\c{c}\~oes externas do sistema. A transforma\c{c}\~ao \'e {\it quase-est\'atica} se as condi\c{c}\~ao externa mudar t\~ao lentamente que em qualquer instante o sistema est\'a aproximadamente em equil\'ibrio. \'E {\it revers\'ivel} se a transforma\c{c}\~ao reconstitui sua hist\'oria no tempo quando a condi\c{c}\~ao externa reconstitui sua hist\'oria no tempo. Uma transforma\c{c}\~ao revers\'ivel \'e quase-est\'atica, mas o contr\'ario n\~ao \'e necessariamente verdade. Por exemplo, um g\'as que se expande livremente em sucessivos elementos infinitesimais de volume experimenta uma transforma\c{c}\~ao quase-est\'atica mas n\~ao revers\'ivel.
\item[(g)] O {\it diagrama} $P$-$V$ de um sistema \'e a proje\c{c}\~ao de uma superf\'icie da equa\c{c}\~ao de estado no plano $P$-$V$. Todo ponto no diagrama $P$-$V$ ent\~ao representa um estado de equil\'ibrio. Uma transforma\c{c}\~ao revers\'ivel \'e um caminho cont\'inuo no diagrama $P$-$V$. Transforma\c{c}\~oes revers\'iveis de tipos espec\'ificos d\~ao origem a caminhos com nomes espec\'ificos, tais como isot\'ermicas, adiab\'aticas, etc. Uma trasnforma\c{c}\~ao que n\~ao \'e revers\'ivel n\~ao pode ser representada.
\item[(h)] O conceito de {\it trabalho} \'e tirado da mec\^anica. Por exemplom oara yn sustena cujos par\^ametros s\~ao $P$, $V$ e $T$, o trabalho $dW$ feito por um sistema numa transforma\c{c}\~ao infinitesimal na qual o volume aumento por $dV$ \'e dado por
\begin{center}
$dW=PdV$
\end{center}
\item[(i)] {\it Calor} \'e o que \'e absorvido por um sistema homog\^eneo se sua temperatura aumente enquanto nenhum trabalho \'e feito. Se $\Delta Q$ \'e uma pequena quantidade de calor absorvido e $\Delta T$ \'e uma pequena mudan\c{c}a na temperatura acompanhando a absor\c{c}\~ao di cakirm a {\it capacidade calor\'ifica} $C$ \'e definida por
\begin{center}
$\Delta Q=C\Delta T$
\end{center}
A capacidade calor\'ifica depende da natureza detalhada do sistema e \'e dada como parte da especifica\c{c}\~ao do sistema. \'E um fato experimental que, para um mesmo $\Delta T$, $\Delta Q$ \'e diferente para formas diferentes de aquecer o sistema. Da mesma maneira, a capacidade calor\'ifica depende da forma de aquecimento. AS capacidade calor\'ificas comumente consideradas s\~ao $C_{V}$ e $C_{P}$, as quais correspondem respectivamente a um aquecimento a $V$ e $P$ constantes. Capacidade calor\'ificas por unidade de massa ou por mol de uma subst\^ancia s\~ao chamados {\it calores espec\'ificos}.
\item[(j)] Um {\it reservat\'orio de calor}, ou simplesmente {\it reservat\'orio}, \'e um sistema t\~ao grande que o ganho ou perda de qualquer quantia finita de calor n\~ao muda sua temperatura.
\item[(k)] Um sistema \'e {\it isolado termicamente} se nenhuma troca de calor pode ocorrer entre ele e o mundo externo. Isolamente t\'ermico pode ser alcan\c{c}ado cercando o sistema com uma {\it parede adiab\'atica}. Qualquer transforma\c{c}\~ao que o sistema possa passar no isolamento t\'ermico \'e dita ocorrida adiabaticamente.
\item[(l)] A quantidade termodin\^amica \'e dita {\it extensiva} se for proporcional a quantidade de subst\^ancia no sistema sob considera\c{c}\~ao e \'e dita {\it intensiva} se for independente da quantidade de subst\^ancia no sistema sob considera\c{c}\~ao. \'E um fato emp\'irico importante que para uma boa aproxima\c{c}\~ao quantidades termodin\^amicas s\~ao ou extensivas ou intensivas.
\item[(m)] O {\it g\'as ideal} \'e um importante sistema termodin\^amico idealizado. Experimentalmente todos os gases se comportam de maneira universal quanto est\~ao suficientemente dilu\'idos. O g\'as ideal \'e uma idealiza\c{c}\~ao desse comportamento limite. Os par\^ametros para um g\'as ideal s\~ao press\~ao $P$, volume $V$, temperatura $T$, e o n\'umero de mol\'eculas $N$. A equa\c{c}\~ao de estado \'e dada pela lei de Boyle:
\begin{eqnarray*}
\frac{PV}{N}=constante ~~~~~~~~~~~~~~~~ (para~temperatura~constante)
\end{eqnarray*}
O valor dessa constante depende da escala experimental utilizada.
\item[(n)] A equa\c{c}\~ao de estado de um g\'as ideal na realidade define uma escala de temperatura, a {\it temperatura do g\'as-ideal} $T$:
\begin{eqnarray*}
PV=NkT
\end{eqnarray*}
onde
\begin{center}
$k=1,38 \times 10^{-16}$ erg/deg
\end{center}
que \'e a chamada constante de Boltzmann. Seu valor \'e determinado por uma escolha convencional dos intervalos de temperatura, chamados, o grau Cent\'igrado. Essa escaka ten yn car\'ater universal porque o g\'as ideal tem um car\'ater universal. A origem $T=0$ \'e arbitrariamente escolhida aqui. Mais tarde n\'os vemos que isto atualmente possui um significado absoluto de acordo com a segunda lei da termodin\^amica.
\end{itemize}

Para construir a escala de temperatura do g\'as-ideal n\'os proceder como se segue. Me\c{c}a $PV/Nk$ de um g\'as ideal na temperatura na qual a \`agua ferve e na qual a \'agua congela. Plote esses dois pontos e desenho uma linha reta de um para o outro, como mostrado na Fig. \ref{figC01_002}. O encontro dessa linha com a abscissa \'e escolhido para ser a origem da escala. Os intervalos da escala de temperatura s\~ao ent\~ao escolhidos tais que existam 100 divis\~oes iguais entre os pontos de ebuli\c{c}\~ao e de congelamento.da \'agua. A escala resultante \'e a escala Kelvin (K). Para usar a escala, trazer qualquer coisa a qual a temperatura \'e pra ser medida en contato t\'ermico com um g\'as ideal (e.g., g\'as h\'elio \`a suficiente densidade baixa), me\c{c}a $PV/Nk$ do g\'as ideal, e leia a temperatura na Fig. \ref{figC01_002}. Uma forma equivalente da equa\c{c}\~ao de estado de um g\'as ideal \'e
\begin{center}
$PV=nRT$
\end{center}
onde $n$ \'e o n\'umero de mols do g\'as e $R$ \'e a constante do g\'as:
\begin{eqnarray*}
R&=&8,135 ~ joule/deg\\
&=&1,986 ~ cal/deg\\
&=&0,0821 ~ litros-atm/deg
\end{eqnarray*}
Seu valor resulta do valor da constante de Boltzmann e do n\'umero de Avogadro:
\begin{center}
n\'umero de Avogadro $=6,023 \times 10^{23}$ \'atomos/mol

\end{center}
\begin{figure}[h]
\begin{center}
%\includegraphics[angle=0,width=0.6\textwidth,clip]{fig002.eps}
\renewcommand{\figurename}{Fig.}
\caption{A escala de temperatura do g\'as-ideal.}
\label{figC01_002}
\end{center}
\end{figure}

A maioria desses conceitos s\~ao entendidos apropriadamente somente em termos moleculares. Aqui n\'os temos que estar satisfeitos com defini\c{c}\~oes emp\'iricas.

Na seq\"u\^encia n\'os introduzimos as leis termodin\^amicas, as quais podem ser vistas como axiomas matem\'aticos definindo um modelo matem\'atico. \'E poss\'ivel deduzir rigorosas conseq\"u\^encias desses axiomas, mais \'e mais importante lembrar que esse modelo n\~ao corresponde rigorosamente ao mundo f\'isico, por ele ignorar a estrutura at\^omica da mat\'eria, e inevitavelmente falhar\'a no dom\'inio at\^omico. No dom\'inio macrosc\'opico, no entanto, termodin\^amica \'e tanto poderosa quanto \'util. Ela permite-nos desenhar conclus\~oes particularmente precisas e inalcans\~aveis de poucas observa\c{c}\~oes aparentemente comuns. Esse poder vem da hip\'otese impl\'icita que a equa\c{c}\~ao de estado \'e uma fun\c{c}\~ao regular, da qual as leis da termodin\^amica, as quais se mostram ing\^enuas numa primeira olhada, est\~ao na realidade bastante limitadas.

\section{A primeira lei da termodin\^amica}
\noindent

Numa transforma\c{c}\~ao termodin\^amica {\it arbitr\'aria} temos $\Delta Q$ denotando uma quantidade l\'iquida de calor absorvido pelo sistema e $\Delta W$ a quantidade l\'iquida de trabalho feito pelo sistema. A primeira lei da termodin\^amica determina que a quantidade $\Delta U$, definida por
\begin{equation}
\Delta U=\Delta Q-\Delta W
\end{equation}
\'e a mesma para todas as transforma\c{c}\~ao levadas de um dado estado inicial para um dado estado final.

Isso define imediatamente uma fun\c{c}\~ao de estado $U$, chamada de energia interna. Seu valor para qualquer estado pode ser encontrado da seguinte maneira. Escolha um estado fixo arbitr\'ario como refer\^encia. Ent\~ao a energia interna de qualquer estado \'e $\Delta Q - \Delta W$ em {\it qualquer} transforma\c{c}\~ao guiadas do estado de refer\^encia para o estado em quest\~ao. Essa \'e definida somente a menos de uma constante arbitr\'aria aditiva. Empiricamente $U$ \'e uma quantidade extensiva. Isso resulta da propriedade de satura\c{c}\~ao das for\c{c}as moleculares, em outras palavras, que a energia de uma subst\^ancia \'e dobrada se sua massa \'e dobrada.

A base experimental da primeira lei \'e a demonstra\c{c}\~ao de Joule da equival\^encia entre calor e energia mec\^anica - a possibilidade de converter trabalho mec\^anico completamente em calor. A inclus\~ao do calor como uma forma de energia leva naturalmente a inclus\~ao do calor no enunciado da conserva\c{c}\~ao de energia. A primeira lei \'e precisamente tal enunciado.

Numa transforma\c{c}\~ao infinitesimal, a primeira lei se reduz ao enunciadoque a diferencial
\begin{equation}
dU=dQ-dW
\end{equation}
\'e exata. Isto \'e, existe uma fun\c{c}\~ao $U$ cujo diferencial \'e $dU$; ou, a integral $\int dU$ \'e independente do caminho de integra\c{c}\~ao e depende somente dos limites de integra\c{c}\~ao. Essa propriedade obviamente n\~ao \'e compartilhada por $dQ$ ou $dW$.