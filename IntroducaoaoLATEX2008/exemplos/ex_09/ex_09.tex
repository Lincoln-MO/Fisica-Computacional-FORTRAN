% Oficina de LaTeX (19.05.2008)
% Autor: Leandro Gutierrez Rizzi
% Exemplo 09

\documentclass[10pt]{article}
\usepackage{a4wide}

\pagestyle{empty}

\begin{document}

O algoritmo utilizado foi proposto em 2001 pelos f\'isicos F. Wang e D. P. Landau \cite{wanglandau}. N\'os nos propomos (...).

\renewcommand{\refname}{Refer\^encias}

\begin{thebibliography}{99}

\bibitem{ibooth} I. Booth, A. B. MacIsaac, J. P. Whitehead, K. De'Bell, Phys Rev. Lett. {\bf 75}, (1995).

\bibitem{cannas} S. A. Cannas, D. A. Stariolo e F. A. Tammarit, Phys. Rev. B {\bf 69}, 092409 (2004).

\bibitem{casartelli} M. Casartelli et al., J. Phys. A: Math. Gen. {\bf 37}, 11731 (2004).

\bibitem{toukmaji} A. Y. Toukmaji e J. A. Board Jr., Comp. Phys. Comm. {\bf 95}, 73-92 (1996).

\bibitem{bergneuhausPLB} B. A. Berg e T. Neuhaus, Phys. Lett. B {\bf 267}, 249 (1991).

\bibitem{bergneuhausPRL} B. A. Berg e T. Neuhaus, Phys. Rev. Lett. {\bf 68}, 9 (1992).

\bibitem{wanglandau} F. Wang e D. P. Landau, Phys. Rev. Lett. {\bf 86}, 2050 (2001).

\bibitem{tables} G. Danese, I. De Lotto, D. Dotti e F. Leporati, Comp. Phys. Commun. {\bf 108}, (1998).

\bibitem{gaozeng} G. T. Gao e X. C. Zeng, J. Chem. Phys. {\bf 106}, (1997).

\bibitem{nijboer} B. R. A. Nijboer e F. W. de Wette, Physica {\bf 23}, 309 (1957).

\bibitem{gradshteyn} I. S. Gradshteyn e I. M. Rhyzik, {\it Tables of Integrals, Series and Products}. Academic, New York (1980).

\bibitem{allen} M. P. Allen e D. J. Tildesley, {\it Computer Simulation of Liquids}. Claredon, Oxford (1989).

\bibitem{binder} K. Binder e D. W. Heerman, {\it Monte Carlo Simulation in Statistical 
Physics, An Introduction}. Springer Series in Solid-State Sciences 80, (1987).

\end{thebibliography}

\end{document}