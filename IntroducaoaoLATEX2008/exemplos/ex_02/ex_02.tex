% Oficina de LaTeX (12.05.2008)
% Autor: Leandro Gutierrez Rizzi
% Exemplo 02

\documentclass[10pt]{article}
\usepackage[T1]{fontenc}
\usepackage[brazil]{babel}
\usepackage[utf8]{inputenc}
\usepackage{a4wide}

\begin{document}
\noindent
Este é o seu {\bf segundo} documento utilizando \LaTeX.\\
Algumas modificações foram
 feitas para estudarmos alguns
pacotes. Agora está bem mais fácil!!!

Mas, relembrando a letra da música "Breathe":

\begin{quote}
(...) quando finalmente o trabalho terminar\\
Não se sente, é hora de cavar um outro.
\end{quote}

Assim, nós temos que cavar outros buracos, e muitos.

\newpage

\begin{center}
Podemos centralizar o texto para mostrar os diversos tamanhos de fontes:

\vspace{0.5cm}

{\Huge Huge}

\vspace{0.5cm}

{\huge huge}

\vspace{0.5cm}

{\Large Large}

\vspace{0.5cm}

{\large large}

\vspace{0.5cm}

{\small small}

\vspace{0.5cm}

{\small tiny}

\end{center}

\begin{flushright}
Ou ainda podemos alinhar o texto à direita para mostrar os tipo de fontes:

\vspace{0.5cm}

{\sffamily Sans Serif}

\vspace{0.5cm}

{\bf {\sffamily Sans Serif bold}}

\vspace{0.5cm}

\texttt{Letra de máquina de escrever}

\vspace{0.5cm}

{\it \texttt{Letra em itálico da máquina de escrever}}

\end{flushright}

% Oficina de LaTeX (12.05.2008)
% Autor: Leandro Gutierrez Rizzi
% Exemplo 01 (+include)

\newpage

\thispagestyle{empty}

\noindent
Muitas vezes nos deparamos com um número mínimo de páginas para escrever sobre um determinado assunto. Quando faltam palavras e não há muito o que dizer, o que fazemos é aumentar o espaçamento entre as linhas, além, é claro, de aumentar o tamanho da fonte.
\\

\baselineskip=18pt

{\large
\noindent
Muitas vezes nos deparamos com um número mínimo de páginas para escrever sobre um determinado assunto. Quando faltam palavras e não há muito o que dizer, o que fazemos é aumentar o espaçamento entre as linhas, além, é claro, de aumentar o tamanho da fonte.\footnote{O espaçamento entre linhas aqui é de 18pt e o tamanho de letra é o ``large''.}.}

\end{document}