% Oficina de LaTeX (19.05.2008)
% Autor: Leandro Gutierrez Rizzi
% Exemplo 06

\documentclass[11pt,leqno]{article}
\usepackage[brazil]{babel}
\usepackage[utf8]{inputenc}
\usepackage[T1]{fontenc}
\usepackage{a4wide}
\usepackage{amsmath}

\baselineskip=14pt

\begin{document}

\newcommand{\somatorio}[1]{
\sum_{n=1}^{\infty} #1
}
\begin{eqnarray*}
\somatorio{x^{n}} \\
\somatorio{\frac{(-1)^n x^{n}}{n!}}
\end{eqnarray*}

\vspace{2.0cm}

\newcommand{\integralimpropria}[2]{
\begin{equation}
\int_{-\infty}^{\infty}#2(#1)d#1
\nonumber
\end{equation}
}

\integralimpropria{r}{g}

\integralimpropria{E}{\Omega_{1}}

\vspace{2.0cm}

\newcommand{\matriz}[2]{
#1= \hbar \left[
\begin{array}{clcr}
#2_{1} &        &        &        \\
       & #2_{2} &        &        \\
       &        & \ddots &        \\
       &        &        & #2_{n}
\end{array}
\right]
}

\begin{equation}
\matriz{M}{\lambda}  ~~~~~~~~ \hbar = \text{constante de Planck}
\nonumber
\end{equation}

\end{document}