% Oficina de LaTeX (12.05.2008)
% Autor: Leandro Gutierrez Rizzi
% Exemplo 04

\documentclass[eps,11pt]{book}
\usepackage[brazil]{babel}
\usepackage[utf8]{inputenc}
\usepackage[T1]{fontenc}
\usepackage{amsmath}
\usepackage{a4wide}
\usepackage{graphicx}

\baselineskip=12pt

\begin{document}
%A Vida, o Universo e Tudo Mais - Douglas Adams
\renewcommand{\chaptername}{Capítulo}

\title{A Vida, o Universo e Tudo Mais}
\author{Douglas Adams}

\maketitle

\newpage 
\thispagestyle{empty}
\flushbottom
Para Sally

\chapter{}

O já habitual grito matinal de horror era o som de Arthur Dent ao acordar e lembrar-se de onde estava.

O que o perturbava não era apenas a caverna fria nem o fato de ser úmida e fedorenta. Era o fato de que ela ficava bem no meio de Islington e que o próximo ônibus só iria passar dentro de dois milhões de anos.

O tempo é, por assim dizer, o pior lugar onde ficar perdido, como Arthur Dent havia descoberto. Ele já tinha se perdido várias vezes, tanto no tempo quanto no espaço. Pelo menos estar perdido no espaço mantém a pessoa ocupada.

Estava ilhado na Terra Pré-Histórica como resultado de uma complexa seqüência de eventos envolvendo o fato de ele ter sido alternadamente detonado ou insultado em regiões da Galáxia mais estranhas do que poderia sonhar. Por conta disso, ainda que sua vida no momento fosse extremamente monótona, continuava se sentindo muito assustado.

Fazia cinco anos que ninguém o detonava.

Como não tinha encontrado ninguém desde que ele e Ford Prefect se separaram quatro anos antes, também não havia sido insultado durante todo aquele tempo.

Exceto uma vez.

Aconteceu numa tarde de primavera cerca de dois anos antes.

Ele estava voltando para sua caverna, pouco depois do entardecer, quando percebeu estranhas luzes piscando através das nuvens. Virou-se para observar, sentindo seu coração encher-se de esperança. Resgate. Uma saída. O sonho impossível de todo náufrago: uma nave.

Observou, fascinado e animado, uma nave prateada e comprida descer em meio à brisa morna da tarde, em silêncio, delicadamente, suas longas e esguias hastes desdobrando-se em um suave bale tecnológico.

Assentou-se suavemente no terreno e o pequeno zumbido que havia gerado sumiu, como se fosse embalado pela calma da tarde.

Uma rampa estendeu-se.

Surgiram luzes pela abertura.

Uma silhueta alta apareceu na portinhola, desceu a rampa e parou bem na frente de Arthur.

-- Você é um idiota, Dent -- foi tudo o que disse.

Era um alienígena, do tipo bem alienígena. Tinha uma altura peculiarmente alienígena, uma cabeça achatada peculiarmente alienígena, pequenos olhos em fenda peculiarmente alienígenas, estava vestido com uma roupa elaboradamente desenhada e usava um colar peculiarmente alienígena, e tinha uma cor pálida cinza-esverdeada de alienígena que reluzia com um brilho lustroso que a maioria das faces cinza-esverdeadas só podia conseguir por meio de muitos exercícios e de sabonetes absurdamente caros.

Arthur olhou-o, atônito.

O alienígena olhou-o de volta.

O sentimento inicial de esperança e excitação havia sido completamente superado pelo espanto, e pensamentos de todos os tipos estavam, naquele momento, brigando pelo controle de suas cordas vocais.

-- Qqqu...? -- disse ele. -- Mmms... ah... aahn... -- acrescentou em seguida. -- Qqqm... eeeerrr... ehh... quem? -- conseguiu finalmente dizer e depois caiu numa espécie de silêncio frenético. Estava sentindo os efeitos de não ter dito nada a ninguém por mais tempo do que podia se lembrar.

A criatura alienígena franziu o rosto brevemente e consultou uma espécie de prancheta que estava segurando com sua mão fina e esguia de alienígena.

-- Arthur Dent? -- disse ele.

Arthur assentiu, balançando a cabeça.

-- Arthur Phillip Dent? -- prosseguiu o alienígena, com um tom de voz de firme.

-- Ahhh... ah... sim... éééé... éééé -- confirmou Arthur.

-- Você é um idiota -- repetiu o alienígena --, um bundão completo.

-- Ehhh...

A criatura pareceu ter ficado satisfeita com aquilo. Balançou a cabeça levemente, depois fez uma marquinha peculiarmente alienígena em sua prancheta e virou-se bruscamente, caminhando em direção à nave.

-- Ehhh... -- disse Arthur, desesperado. -- Ehhhh...

-- Ah, não me venha com esse papo! -- retrucou o alienígena. Subiu a rampa, passou pela portinhola e desapareceu dentro da nave. A portinhola se fechou, a rampa foi recolhida e a nave começou a emitir um leve zumbido grave.

-- Ehhh, hei! -- gritou Arthur, correndo logo em seguida na direção da nave. -- Espere aí! -- disse. -- O que foi isso? O quê? Espere!

A nave elevou-se no ar, removendo seu peso como quem joga uma capa no chão, e pairou brevemente. Balançava estranhamente no céu da tarde. Passou pelas nuvens, iluminando-as brevemente, e depois se foi, deixando Arthur sozinho, naquela imensidão de terra, dançando uma pequena dança patética e sem sentido.

-- O quê? -- gritou Arthur. -- O quê? Quê? Ei, o que foi? Volte aqui e repita isso!

Pulou e dançou até suas pernas começarem a tremer, gritou até seus pulmões arderem. Ninguém respondeu. Não tinha ninguém para ouvir ou falar com ele.

A nave alienígena já cruzava em alta velocidade as camadas mais altas da atmosfera, a caminho do vazio aterrador que separa as poucas coisas que existem no Universo umas das outras.

No interior da nave, seu ocupante, o alienígena com a pele milionária, estava esticado no único assento. Seu nome era Wowbagger, o Infinitamente Prolongado. Um homem com um objetivo. Na verdade, não era um objetivo muito nobre, como ele mesmo seria o primeiro a admitir, mas ao menos tinha um objetivo e isso o mantinha ocupado.

Wowbagger, o Infinitamente Prolongado, era -- na verdade, é -- um dos pouquíssimos seres imortais do Universo.

Aqueles que já nascem imortais sabem como lidar com isso instintivamente.

Contudo, Wowbagger não tinha nascido imortal. Não. Passou a desprezar os imortais, aquela corja de babacas tranqüilões. Tinha se tornado imortal por um infeliz acidente envolvendo um acelerador de partículas irracionais, uma refeição líquida e um par de elásticos. Os detalhes exatos do acidente não são importantes, porque ninguém jamais foi capaz de duplicar as circunstâncias exatas em que as coisas aconteceram e, ao tentarem, muitas pessoas acabaram ficando com cara de idiotas, morreram no processo, ou ambas as coisas.
Com uma careta e uma expressão de cansaço, Wowbagger fechou seus olhos, colocou uma música de fundo no som da nave e pensou que até poderia ter conseguido... Se não fosse pelas tardes de domingo, teria conseguido.

No início tudo parecia engraçado: havia se divertido muito, vivendo perigosamente, se arriscando ao extremo, enriquecendo com investimentos de longo prazo e altas taxas de retorno e, no geral, permanecendo vivo enquanto os outros morriam.

Contudo, no final foram as tardes de domingo que se tornaram insuportáveis: aquela terrível sensação de não ter absolutamente nada para fazer que se instala em torno das 14h55, quando você sabe que já tomou um número mais que razoável de banhos naquele dia, quando sabe que, por mais que tente se concentrar nos artigos dos jornais, você nunca conseguirá lê-los nem colocar em prática a nova e revolucionária técnica de jardinagem que eles descrevem, e quando sabe que, enquanto olha para o relógio, os ponteiros se movem impiedosamente em direção às 16 horas e logo você entrará no longo e sombrio entardecer da alma.

A partir daí as coisas começaram a perder o sentido. Os sorrisos alegres que costumava distribuir durante os funerais dos outros começaram a sumir. Aos poucos, começou a desprezar o Universo em geral e cada um dos seus habitantes em particular.

Foi então que concebeu seu objetivo, aquilo que o faria prosseguir e que, até onde podia compreender, iria fazê-lo prosseguir para todo o sempre. Era o seguinte:

Iria insultar o Universo.

Isto é, iria insultar todos no Universo. Individualmente, pessoalmente e -- esse foi o ponto no qual realmente decidiu se empenhar -- em ordem alfabética.

Quando as pessoas reclamavam cora ele, como algumas vezes já o tinham feito, que o plano não somente era mal-intencionado como também completamente impossível, devido ao número de pessoas que nasciam e morriam sem parar, ele simplesmente as encarava com um olhar gélido e dizia:

-- Um homem tem o direito de sonhar, não é?

Foi assim que tudo começou. Construiu uma nave feita Para durar, com um computador capaz de lidar com a infinitude de dados necessário para manter o controle de toda a população do Universo conhecido e calcular as complicadas rotas envolvidas.

Sua nave atravessou as órbitas internas do sistema estelar Sol, preparando-se para ganhar impulso ao circundar sua estrela e depois partir para o espaço interestelar.

-- Computador.

-- Presente -- respondeu o computador.

-- Para onde vamos?

-- Vou calcular.

Wowbagger observou por alguns instantes o intricado colar de brilhantes da noite, bilhões de pequenos diamantes polvilhando a infinita escuridão com sua luz. Cada um deles, absolutamente todos, estava em seu itinerário. Iria passar milhões de vezes pela grande maioria deles.

Imaginou brevemente sua rota, conectando todos os pontos do céu como um desenho infantil de unir os pontos. Torceu para que, visto de algum lugar do Universo, aquele traçado soletrasse uma palavra extremamente obscena.

O computador emitiu um bipe chocho para indicar que havia terminado seus cálculos.

-- Folfanga -- disse. E bipou novamente. -- Quarto planeta do sistema Folfanga -- prosseguiu. E bipou mais uma vez. -- Duração estimada para a viagem: três semanas -- disse depois. Bipou de novo. -- Vamos encontrar uma pequena lesma -- bipou -- do gênero A-Rth-Urp-Hil-Ipdenu. -- Acredito -- acrescentou, após uma breve pausa na qual bipou -- que você decidiu chamá-la de "bundona descerebrada".

Wowbagger resmungou. De sua janela, observou a grandiosidade da criação por mais alguns instantes.

-- Acho que vou tirar um cochilo. Por quais redes de transmissão vamos passar durante as próximas horas?

O computador bipou.

-- Cosmovid, Thinkpix e Home Brain Box -- disse. Então bipou mais uma vez.

-- Vai passar algum filme a que eu ainda não tenha assistido umas 30 mil vezes?

-- Não.

-- Ah.

-- Bem, tem Angústia no Espaço. Este você só viu 33.517 vezes.

-- Me acorde para a segunda parte. O computador bipou.

-- Durma bem -- disse.

A nave deslizava pela noite.

Enquanto isso, na Terra, caía uma chuva fina. Arthur Dent sentou-se em sua caverna e teve uma das noites mais tenebrosas de sua vida, pensando em milhares de coisas que poderia ter dito ao alienígena e matando mosquitos, que também tiveram uma noite bem tenebrosa.
No dia seguinte, decidiu fazer uma sacola usando uma pele de coelho porque achou que seria útil para colocar coisas dentro.

\chapter{}

Dois anos depois disso ter acontecido, a manhã estava doce e calma quando Arthur saiu da caverna que chamava de "casa" até conseguir encontrar um nome melhor para aquilo ou então encontrar uma caverna melhor.

Sua garganta estava novamente irritada devido a seu grito matinal de horror, mas ainda assim ele estava de ótimo humor. Enrolou firmemente seu roupão esfarrapado ao redor do corpo e sorriu, feliz, olhando aquela linda manhã.

O ar estava claro e cheio de aromas suaves, a brisa acariciava levemente a grama alta que cercava a caverna, os pássaros gorjeavam uns para os outros, as borboletas borboleteavam lindamente ao seu redor e toda a natureza parecia conspirar para ser tão gentil e agradável quanto possível.

Não eram, contudo, aquelas delícias bucólicas que haviam deixado Arthur tão feliz. Ele acabara de ter uma ótima idéia sobre como lidar com o terrível e solitário isolamento, os pesadelos, o fracasso de todas as suas tentativas de horticultura e a completa ausência de futuro e a futilidade de sua vida ali, na Terra pré-histórica. Tinha decidido enlouquecer.
Sorriu de novo, feliz, e mordeu um pedaço de perna de coelho que havia sobrado de seu jantar. Mastigou alegremente durante algum tempo e então resolveu anunciar formalmente sua decisão.

Ficou de pé, endireitou o corpo e olhou de frente para os campos e montanhas. Para dar mais peso às suas palavras, enfiou o osso de coelho na barba. Abriu bem os braços e disse:

-- Vou ficar louco!
\end{document}